\documentclass[11pt]{article}

\setlength{\parindent}{0pt}
\usepackage{xltxtra}
\usepackage{hyperref}
\hypersetup{hidelinks}
\usepackage{url}
\urlstyle{tt}
\usepackage{xcolor}
\definecolor{CVBlue}{RGB}{23,110,191}
\usepackage{calc}
\usepackage{graphicx}
\usepackage{tikz}
\usetikzlibrary{calc}
\usepackage{fontspec}
\usepackage{xeCJK}
\usepackage{enumitem}
\CJKsetecglue{} %% 取消中文与数字之间的间隙

%% 主文档字体设置
\setmainfont[
    Path = fonts/Main/,
    Extension = .otf,
    BoldFont = texgyretermes-bold.otf, % 加粗字体
]{texgyretermes-regular.otf} % 正文字体

% 中文字体设置
\setCJKmainfont[
    Path = fonts/hansans/,
    Extension = .ttf,
    BoldFont = NotoSansSC-Bold.ttf, % 加粗字体
]{NotoSansSC-Regular.otf} % 正文字体

\usepackage{relsize}
\usepackage{xspace}

% 使用 fontawesome(部分图标)
\usepackage{fontawesome} 

% A4纸,上下左右边距
\usepackage[
    a4paper,
    left=1.2cm,
    right=1.2cm,
    top=1.5cm,
    bottom=1cm,
    nohead
]{geometry}

\renewcommand{\baselinestretch}{1.5} % 行间距设为1.5

\usepackage{titlesec}
\usepackage{enumitem}
\setlist{noitemsep} % 取消列表项间的额外间距
\setlist[itemize]{topsep=0.25em, leftmargin=*}
\setlist[enumerate]{topsep=0.25em, leftmargin=*}

% --- 用于控制【不同项目之间】的垂直距离 ---
\newlength{\interProjectSpacing}
\setlength{\interProjectSpacing}{0.9em} % <--- 在此调整项目之间的距离
\newcommand{\projectsep}{\vspace{\interProjectSpacing}}

% --- 用于控制【项目标题】与下方【项目描述】的距离 ---
\newlength{\intraProjectTitleSep}
\setlength{\intraProjectTitleSep}{0.4em} % <--- 在此调整标题和描述的距离
\newcommand{\titlebreak}{\\[\intraProjectTitleSep]}

% --- 用于控制【项目描述】与下方【要点列表】的距离 ---
\newlength{\intraProjectListTopSep}
\setlength{\intraProjectListTopSep}{0.2em} % <--- 在此调整描述和列表的距离

% =======================================================================

\titleformat{\section}         % 定制 \section 命令 
{\large\bfseries\raggedright} % 将 section 标题设置为大号、粗体且左对齐
{}{0em}                      % 可用于为所有 section 添加前缀(如“章节...”)
{}                           % 可用于在标题前插入代码
[{\color{CVBlue}\titlerule}]  % 在标题后插入一条横线
\titlespacing*{\section}{0cm}{*1.6}{*1.2}

\begin{document}
\pagenumbering{gobble}

\centerline{\LARGE\bfseries{Boyan Zhang}} 

\centerline{\normalsize{\faPhone\ (+46) 764581398 \quad \faEnvelopeO\ \href{mailto:boyan-z@outlook.com}{boyan-z@outlook.com}}} 

\centerline{\normalsize{\faGithubSquare\ \href{https://github.com/boyanzhang}{https://github.com/boyanzhang} \quad \faRssSquare\ \href{https://boyanzhang.github.io/}{https://boyanzhang.github.io/}}}

\section{\makebox[\widthof{\faGraduationCap}][c]{\color{CVBlue}\faGraduationCap}\ Education}
\textbf{KTH Royal Institute of Technology} \hfill Aug 2025 – Present\\[0.5em] 
M.Sc. in Computer Science
\begin{itemize}[nosep]
    \item Relevant courses: Machine Learning, Natural Language Processing, Computer Networks, Database Systems
\end{itemize}

\textbf{Zhejiang University} \hfill Aug 2022 – Jun 2025\\[0.5em]
B.Eng. in Computer Science and Technology, GPA: 3.99/4.3
\begin{itemize}[nosep]
    \item Relevant courses: Programming, Machine Learning, Computer Networks
\end{itemize}

\section{\makebox[\widthof{\faUsers}][c]{\color{CVBlue}\faUsers}\ Project Experience}

\textbf{Mini-OS} \hfill Aug 2024 – Oct 2024 \titlebreak
Developed a minimalist Linux-like OS for RV64 with file I/O and process management using C and RISC-V assembly. Strengthened understanding of Linux internals and low-level system design.

\textbf{E-Commerce Price Comparison Platform} \hfill Jan 2025 – Mar 2025 \titlebreak
Developed a full-stack web application for product price comparison across multiple e-commerce sites using Java, Spring Boot, and Vue.js.

\projectsep

\textbf{Mobile Developer Intern, ErholCare} \hfill Jun 2024 – Aug 2024 \titlebreak
ErholCare is a comprehensive platform for care workers and healthcare organizations to manage documentation, schedules, and team collaboration efficiently.
\begin{itemize}[nosep]
    \item Developed mobile application using Dart and Flutter for ErholCare, an AI-driven platform.
    \item Enhanced mobile app features for documentation management, scheduling, and team collaboration.
    \item Optimized app performance and user experience in an agile development environment.
\end{itemize}

\section{\makebox[\widthof{\faCogs}][c]{\color{CVBlue}\faCogs}\ Skills}
\begin{itemize}[nosep]
    \item \textbf{Programming Languages:} Dart, Flutter, C/C++, Java, Python
    \item \textbf{Tools & Frameworks:} PyTorch, Spring Boot, MySQL, Git, Xilinx Vivado
\end{itemize}

\section{\makebox[\widthof{\faGraduationCap}][c]{\color{CVBlue}\faList}\ Awards}
\begin{itemize}
    \item National Encouragement Scholarship (China) \hfill 2024
\end{itemize}

\section{\makebox[\widthof{\faInfo}][c]{\color{CVBlue}\faInfo}\ Other}
\begin{itemize}[parsep=0.5ex]
    \item \textbf{Technical Blog:} \href{https://boyanzhang.github.io/}{https://boyanzhang.github.io/}
    \item \textbf{GitHub:} \href{https://github.com/boyanzhang}{https://github.com/boyanzhang}
    \item \textbf{Languages:} Fluent in English and Mandarin, CET-4, CET-6
\end{itemize}

\end{document}
